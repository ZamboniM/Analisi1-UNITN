\chapter{Insiemistica}
Un insieme \`e una collezione di oggetti i quali sono detti elementi dell'insieme. Un insieme \`e definito se \`e possibile determinare univocamente se un elemento appartiene o no all'insieme. Un insieme viene genericamente indicato con una lettera un stampato maiuscolo, mentre i suoi elementi con lettere minuscole. Se \emph{x} \`e un elemento di E, \emph{x} appartiene ad E, $\emph{x} \in E$, se \emph{x} non appartiene ad E  $\emph{x} \not\in E$.Gli insiemi si possono definire per enumerazione, elencando cio\`e ogni elemento in essi presente o determinando una caratteristica che accomuna tutti gli elementi di tale insieme. L'insieme privo di elementi \`e l'insieme vuoto ($\{\},\emptyset$). Un sottoinsieme \`e un insieme in cui ogni elemento \`e contenuto nell'insieme di cui \`e sottoinsieme. 
\section{Convenzioni di scrittura degli insiemi}
\begin{center}
$E=\{x:\emph{P}(x)\} \wedge \emph{P}(x) = \emph{Q}(x)\wedge\emph{S}(x)\wedge\emph{T}(x) \Rightarrow E=\{x:\emph{Q}(x),\emph{S}(x),\emph{T}(x)\}$\\
$E=\{x:x \in \emph{P}(x)\wedge\emph{F}(x)\} \Rightarrow  E=\{x\in\emph{P}(x):\emph{F}(x)\}$\\
\end{center}
\section{Operazioni tra insiemi}
\subsection{Unione di insiemi}
\begin{equation}
F(E \cup U) = \{ x: x\in E\lor x\in U\}
\end{equation}

\subsection{Intersezione di insiemi}
\begin{equation}
F(E \cap U) = \{ x: x\in E\wedge x\in U\}
\end{equation}

\subsection{Differenza di insiemi}
\begin{equation}
E \backslash F =\{x:x\in E, x\not\in F\}
\end{equation}

\subsection{Complementare di un insieme E rispetto all'insieme universale X}
\begin{equation}
X\backslash E=E^{c}=\{x\in X: x\not\in E\}
\end{equation}
\subsection{Alcune propriet\`a di unione e intersezione}
Le due operazioni sono commutative, associative e distributrive.
\section{Caratterizzazione di insiemi}
Se $E \cap U = \emptyset$ E ed F si dicono disgiunti.
L'insieme delle parti di un insieme, P(E), \`e l'insieme che contiene tutti i sottoinsiemi dell'insieme di partenza, ha cardinalit\`a $2^{n}$, dove n \`e la cardinalit\`a
dell'insieme di partenza.\\
Prodotto cartesiano di E e F: $E\times F=\{(x,y):x\in E, y\in F\}$, $E\times E=E^2$, $R\times R=R^2$ viene rappresentato graficamente con il piano cartesiano.
\section{Leggi di De Morgan}
\begin{gather*}
E,F \subset X:\\
(E \cup F)^{c}=E^{c}\cap F^{c}\\
(E \cap F)^{c}=E^{c}\cup F^{c}
\end{gather*}
