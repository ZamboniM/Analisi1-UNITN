\chapter{Derivate}
\section{Definizione}
Sia $I \subseteq \mathbb{R}$ intercallo; $f: I \to \mathbb{R}$; $x_0 \in I$. Chiameremo \textbf{derivata} di $f$ in $x_0$:
\begin{equation}
f^{\prime}(x_0) = \diff{f(x_0)}{x} \triangleq \displaystyle \lim_{x \to x_0} \frac{f(x) - f(x_0)}{x - x_0} = \lim_{h \to 0} \frac{f(x +h) - f(x_0)}{h}
\end{equation}
se il limite esiste (sia finito che infinito).\\
$\frac{f(x) - f(x_0)}{x - x_0}$ è chiamato rapporto incrementale

\section{Retta tangente}
Viene definita retta tangente $f$ in $x_0$ la retta di equazione:
\begin{equation}
y = f^{\prime}(x_0)(x - x_0) + f(x_0)
\end{equation}
Il rapporto incrementale è quindi il coefficente angolare della retta tangente

\section{Derivabilità}
La funzione $f$ si dice derivabile in $x_0$ se $\exists\ f^{\prime}(x_0)$ ed è finito
\subsection{Osservazione}
\begin{equation}
f \text{ derivabile in } x_0 \implies f \text{ continua in } x_0
\end{equation}
Da porre particolare attenzione al fatto che non vale il contrario

\section{Errore}
\begin{equation}
E_1 \triangleq f(x) - (f(x_0) + f^{\prime}(x - x_0)) = o(x - x_0) \text{ per } x \to x_0
\end{equation}
L'errore di approssimazione della funzione alla retta tangente in $x_0$ è quindi un o-piccolo di $(x - x_0)$
\subsection{Dimostrazione}
\begin{equation}
\displaystyle \lim_{x \to x_0} \frac{f(x) - f(x_0) - f^{\prime}(x - x_0)}{x - x_0} = \lim_{x \to x_0} \frac{f(x) - f(x_0)}{x - x_0} - f^{\prime}(x_0) = 0
\end{equation}

\section{Derivata destra e sinistra}
\subsection{Derivata destra}
\begin{equation}
f^{\prime}_+(x_0) \triangleq \displaystyle \lim_{x \to x_0^+} \frac{f(x) - f(x_0)}{x - x_0^-}
\end{equation}
\subsection{Derivata sinistra}
\begin{equation}
f^{\prime}_-(x_0) \triangleq \displaystyle \lim_{x \to x_0} \frac{f(x) - f(x_0)}{x - x_0}
\end{equation}
\subsection{Osservazione}
\begin{equation}
\exists\ f^{\prime}(x_0) \implies \exists\ f^{\prime}_+(x_0) = f^{\prime}(x_0);\ f^{\prime}_-(x_0) = f^{\prime}(x_0)
\end{equation}

\section{Lemma segno $f - f^{\prime}$}
Sia $f: (a,b)  \to \mathbb{R}$ derivabile in $x_0$ con $f(x_0) \neq 0$ allora $f(x) - f(x_0)$ cambia segno attraversando $x_0$.

\section{Punti di non derivabilità}
Sia $f: ]a,b[  \to \mathbb{R}$ continua in $x_0$.\\
Supposto $\exists\ f^{\prime}_-(x_0),\ f^{\prime}_+(x_0)$ e $f^{\prime}_-(x_0) \neq f^{\prime}_+(x_0)$ allora il punto $(x_0, f(x_0)) \in \ ]a,b[ \times \mathbb{R}$ è:
\begin{enumerate}
\item[i)] \textbf{punto angoloso} se $f^{\prime}_-(x_0),\ f^{\prime}_+(x_0)$ o entrambe sono finite
\item[ii)] \textbf{cuspide} se $f^{\prime}_-(x_0) = +\infty$ e $f^{\prime}_+(x_0) = -\infty$ o viceversa
\item[iii)] \textbf{tangente verticale} se $f^{\prime}_-(x_0) = f^{\prime}_+(x_0) = +\infty$ o $-\infty$
\end{enumerate}

\section{Algebretta delle derivate}
\begin{enumerate}
\item[•] $(f+g)^{\prime} = f^{\prime} + g^{\prime}$
\item[•] $(fg)^{\prime} = f^{\prime}g + fg^{\prime}$
\item[•] $(\frac{f}{g})^{\prime} = \frac{f^{\prime}g + fg^{\prime}}{g^2}$
\item[•] $(\frac{1}{f})^{\prime} = -\frac{f^{\prime}}{f^2}$
\item[•] $(f \circ g)^{\prime}(x_0) = f^{\prime}(g(x_0))g^{\prime}(x_0)$\\
$(f \circ g \circ h)^{\prime}(x_0) = f^{\prime}(g(h(x_0)))g^{\prime}(h(x_0))h^{\prime}(x_0)$
\item[•] $(f(x)^{g(x)})^{\prime} = f(x)^{g(x)}[g^{\prime}log(f(x)) + g(x)f^{\prime}(x)]$
\end{enumerate}

\section{Teorema esistenza derivata da derivate destra e sinistra}
Sia $f: ]a,b[$ continua; $x_0 \in ]a,b[$ Supposta $f$ derivabile in ogni $x \neq x_0$:
\begin{enumerate}
\item[i)] se $\exists$no finiti $f^{\prime}_+(x_0) = f^{\prime}_-(x_0) = a$ allora $\exists\ f^{\prime}(x_0) = a$
\item[ii)] se $\exists$no finiti $f^{\prime}_+(x_0) \neq f^{\prime}_-(x_0)$ allora $\not\exists\ f^{\prime}(x_0) = a$
\end{enumerate}

\section{Derivata funzione inversa}
$I$ intervallo, $x_0 \in I; f:I \to \mathbb{R}$ continua e strettamente monotona.\\
Se $f$ è derivabile in $x_0$ e $f^{\prime}(x_0) \neq 0$ allora la funzione $f^{-1}: f(I) \to I$ è derivabile in $f(x_0)$:
\begin{equation}
(f^{-1})^{\prime}(f(x_0)) = \frac{1}{f^{\prime}}
\end{equation}
\subsection{Dimostrazione}
Sia $y_0 = f(x_0);\ y = f(x)$ (ovvero $f^{-1}(y_0) = x_0;\ f^{-1}(y) = x$.
\begin{equation} \displaystyle \begin{split}
(f^{-1})^{\prime}(y_0) = \lim_{y \to y_0}\frac{f^{-1}(y) - f^{-1}(y_0)}{y - y_0} = \lim_{x \to x_0}\frac{f^{-1}(f(x)) - f^{-1}(f(x_0))}{f(x) - f(x_0)}\\
 = \lim_{x \to x_0}\frac{x - x_0}{f(x) - f(x_0)} = \frac{1}{f^{\prime}(x_0)}
\end{split} \end{equation}

\section{Estremi locali di funzione}
\subsection{Teorema di Fermat}
Sia $f: (a,b) \to \mathbb{R};\ x_0 \in (a,b)$. Se $f$ è derivabile in $x_0$ e se $x_0$ è un punto di massimo (o minimo) locale per $f$ allora $f^{\prime}(x_0) = 0$
\subsection{Dimostrazione}
$x_0$ punto di massimo locale $\implies \exists\ U$ intorno di $x_0\ |\ f(x_0) \geq f(x)\ \forall\ x \in U \cap (a,b)$.
\begin{equation} 
\begin{cases}
\displaystyle \lim_{x \to x_0^+}\frac{f(x) - f(x_0)}{x - x_0} \leq 0\\
\displaystyle \lim_{x \to x_0^-}\frac{f(x) - f(x_0)}{x - x_0} \geq 0\\
\exists\ f^{\prime}(x_0) = f^{\prime}_-(x_0) = f^{\prime}_+(x_0)
\end{cases}
\implies f^{-1}(x_0) = 0
\end{equation}
\subsection{Definizione punto critico}
$x_0 \in (a,b)\ |\ f$ è derivabile in $x_0$ e $f^{\prime}(x_0) = 0$ viene chiamato punto critico (o stazionario) di $f$
\subsection{Corollario}
$f: [a,b] \to \mathbb{R};\ x_0$ estremo locale di $f$: Allora $x_0$ è:
\begin{enumerate}
\item[1)] punto critico
\item[2)] appartenente a $(a,b)$
\item[3)] punto di non derivalbilità
\end{enumerate} 

\section{Teorema di Lagrange}
Sia $f: [a,b]$ continua in $[a,b]$, derivabile in $]a,b[$ allora:
\begin{equation}
\exists\ c \in\ ]a,b[\ |\ f^{\prime}(c) = \frac{f(b) - f(a)}{b - a}
\end{equation}
\subsection{Teorema di Rolle}
Sia $f: [a,b]$ continua in $[a,b]$, derivabile in $]a,b[$ e $f(a) = f(b)$ allora:
\begin{equation}
\exists\ c \in\ ]a,b[\ |\ f^{\prime}(c) = 0
\end{equation}
\subsection{Dimostrazione Rolle}
Essendo $f$ continua nell'intervallo chiuso e limitato $[a,b]$; allora $\exists\ \displaystyle \max_{[a,b]}(f), \min_{[a,b]}(f)$
\begin{enumerate}
\item[Caso 1:] $\displaystyle \max_{[a,b]}(f) \text{ e } \min_{[a,b]}(f) \text{ sono assunti agli estremi } \implies \max_{[a,b]}(f) = \min_{[a,b]}(f) = f(a) = f(b) \implies f \text{ costante } \implies f^{\prime} = 0$
\item[Caso 2:] $\displaystyle \max_{[a,b]}(f) \text{ o } \min_{[a,b]}(f)$ all'interno dell'intervallo. Supposto il massimo o il minimo interno $x_0 \in\ ]a,b[ \implies f^{\prime}(x_0) = 0$ per il teorema di Fermat
\end{enumerate}
\subsection{Dimostrazione Lagrange}
Sia $h(x) = f(x) - \biggl [\frac{f(b) - f(a)}{b - a}(x-a) + f(a) \biggl ]$. Essendo $h(a) = h(b) = 0$ per il teorema di Rolle $\exists\ c\ |\ h^{\prime}(c) = 0$. Ma $h^{\prime} = f^{\prime}(x) - \frac{f(b) - f(a)}{b - a} \implies f^{\prime}(c) = \frac{f(b) - f(a)}{b - a}$

\section{Teorema di Cauchy}
Siano $f,g: [a,b] \to \mathbb{R}$ continue in $[a,b]$ e derivabili in $(a,b)$. Allora $\exists\ c \in\ ]a,b[\ |\ [f(b) - f(a)]g^{\prime}(c) = [g(b) - g(a)]f^{\prime}(c) $
\subsection{Osservazione}
Se $g(b) \neq g(a)$ e $g^{\prime}(c) \neq 0$: la tesi è equivalentemente\\
$\frac{f^{\prime}(c)}{g^{\prime}(c)} = \frac{f(b) - f(a)}{b - a}$

\section{Conseguenze Lagrange}
\subsection{Derivata funzione costante}
Sia $f$ derivabile in $I$ e $f^{\prime}(x) = 0\ \forall\ x \in I \implies f$ costante.
\subsubsection{Dimostrazione}
Siano $\forall\ x_1, x_2 \in I;\ x_1 < x_2$. Applicando Lagrange ad $[x_1;x_2]\ \exists\ c \in ]x_1;x_2[\ |\ f^{\prime}(c) = \frac{f(x_2) - f(x_1)}{x_2 - x_1}$ ma essendo per ipotesi $f^{\prime}(c) = 0 \implies f(x_1) = f(x_2)$

\subsection{Test di monotonia}
Sia $f:]a,b[ \to \mathbb{R}$ derivabile Se
\begin{enumerate}
\item[i)] $f^{\prime}(x) \geq 0\ \forall\ x \in (a,b) \iff f$ crescente in $]a,b[$\\
$f^{\prime}(x) \leq 0\ \forall\ x \in (a,b) \iff f$ strettamente decrescente in $]a,b[$
\item[ii)] $f^{\prime}(x) > 0\ \forall\ x \in (a,b) \implies f$ crescente in $]a,b[$\\
$f^{\prime}(x) < 0\ \forall\ x \in (a,b) \implies f$ strettamente decrescente in $]a,b[$
\end{enumerate}

\subsubsection{Dimostrazione punto i)}
Parte $f$ crescente $\implies f^{\prime}(x) \geq 0$:\\
$\forall\ x_1,x_2 \in (a,b);\ x_2>x_1$ per ipotesi $\frac{f(x_2) - f(x_1)}{x_2 - x_1} \geq 0$; Ma $\displaystyle \lim_{x_2 \to x_1}\frac{f(x_2) - f(x_1)}{x_2 - x_1} = f^{\prime}(x_1) \geq 0$\\
\\
Parte  $f^{\prime}(x) \geq 0 \implies$ $f$ crescente:\\ 
Siano $a < x_1 < x_2 < b$. Applicando Lagrange ad $[x_1,x_2]$ si ha che $\exists\ c \in ]x_1,x_2[\ |\ f^{\prime}(c) = \frac{f(x_2) - f(x_1)}{x_2 - x_1} \implies f^{\prime}(c)(x_2- x_1) = f(x_2) - f(x_1)$. Ma essendo $f^{\prime}(c) > 0$ per ipotesi; $(x_2 + x_1) > 0$ per definizione allora anche $(f(x_2) - f(x_1)) > 0$.

\section{Teorema de l'Hopital}
Siano $f,g: (a,b) \to \mathbb{R}$ derivabili in $(a,b)$ e tali che:
\begin{enumerate}
\item[1)] $\displaystyle \lim_{x \to a^+}f(x) = \lim_{x \to a^+}g(x) = 0$
\item[2)] $g^{\prime}(x) \neq 0$ in un intorno di $a$
\item[3)] $\displaystyle \lim_{x \to a^+}\frac{f^{\prime}(x)}{g^{\prime}(x)} = l \in \mathbb{\overline{R}}$
\end{enumerate}
Allora $\exists\ \displaystyle \lim_{x \to a^+}\frac{f(x)}{g(x)} = l$
\subsection{Dimostrazione}
\begin{enumerate}
\item[i)] $g(x) \neq 0$ in un intorno di $a$ in quanto per ipotesi $g^{\prime}(x) \neq 0$ in un intorno di $a \implies g$ strettamente monotona e in quanto (per ipotesi) $\displaystyle \lim_{x \to a^+}g(x) = 0$ si ha che per forza $g(x) \neq 0$ in un intorno di $a$
\item[ii)] Il valore è effettivamente $l$ in quanto se si considerano $f,g$ estese in questo modo:
$f(x) = \begin{cases}
f(x) \text{ per } x>a\\
0 \text{ per } x=a
\end{cases}$;
$g(x) = \begin{cases}
g(x) \text{ per } x>a\\
0 \text{ per } x=a
\end{cases}$. Ottengo due funzioni continue in $[a,x]$ e derivabili in $(a,x)$. Posso quindi applicare Cauchy a $f$ e $g$.\\
$\exists c_x \in (a,x)\ |\ \frac{f(x) - f(a)}{g(x) - g(a)} = \frac{f^{\prime}(c_x)}{g^{\prime}(c_x)} \xrightarrow[x \to a^+;\ c_x \to a^+]{} l$ per ipotesi. Ma $\frac{f(x) - f(a)}{g(x) - g(a)} = \frac{f(x)}{g(x)}$ in quanto $f(a) = g(a) = 0$
\end{enumerate}