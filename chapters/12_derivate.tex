\chapter{Derivate}
\section{Definizione}
Sia $I \subseteq \mathbb{R}$ intercallo; $f: I \to \mathbb{R}$; $x_0 \in I$. Chiameremo \textbf{derivata} di $f$ in $x_0$:
\begin{equation}
f^{\prime}(x_0) = \diff{f(x_0)}{x} \triangleq \displaystyle \lim_{x \to x_0} \frac{f(x) - f(x_0)}{x - x_0} = \lim_{h \to 0} \frac{f(x +h) - f(x_0)}{h}
\end{equation}
se il limite esiste (sia finito che infinito).\\
$\frac{f(x) - f(x_0)}{x - x_0}$ è chiamato rapporto incrementale

\section{Retta tangente}
Viene definita retta tangente $f$ in $x_0$ la retta di equazione:
\begin{equation}
y = f^{\prime}(x_0)(x - x_0) + f(x_0)
\end{equation}
Il rapporto incrementale è quindi il coefficente angolare della retta tangente

\section{Derivabilità}
La funzione $f$ si dice derivabile in $x_0$ se $\exists\ f^{\prime}(x_0)$ ed è finito
\subsection{Osservazione}
\begin{equation}
f \text{ derivabile in } x_0 \implies f \text{ continua in } x_0
\end{equation}
Da porre particolare attenzione al fatto che non vale il contrario

\section{Errore}
\begin{equation}
E_1 \triangleq f(x) - (f(x_0) + f^{\prime}(x - x_0)) = o(x - x_0) \text{ per } x \to x_0
\end{equation}
L'errore di approssimazione della funzione alla retta tangente in $x_0$ è quindi un o-piccolo di $(x - x_0)$
\subsection{Dimostrazione}
\begin{equation}
\displaystyle \lim_{x \to x_0} \frac{f(x) - f(x_0) - f^{\prime}(x - x_0)}{x - x_0} = \lim_{x \to x_0} \frac{f(x) - f(x_0)}{x - x_0} - f^{\prime}(x_0) = 0
\end{equation}

\section{Derivata destra e sinistra}
\subsection{Derivata destra}
\begin{equation}
f^{\prime}_+(x_0) \triangleq \displaystyle \lim_{x \to x_0^+} \frac{f(x) - f(x_0)}{x - x_0^-}
\end{equation}
\subsection{Derivata sinistra}
\begin{equation}
f^{\prime}_-(x_0) \triangleq \displaystyle \lim_{x \to x_0} \frac{f(x) - f(x_0)}{x - x_0}
\end{equation}
\subsection{Osservazione}
\begin{equation}
\exists\ f^{\prime}(x_0) \implies \exists\ f^{\prime}_+(x_0) = f^{\prime}(x_0);\ f^{\prime}_-(x_0) = f^{\prime}(x_0)
\end{equation}

\section{Lemma segno $f - f^{\prime}$}
Sia $f: (a,b)  \to \mathbb{R}$ derivabile in $x_0$ con $f(x_0) \neq 0$ allora $f(x) - f(x_0)$ cambia segno attraversando $x_0$.

\section{Punti di non derivabilità}
Sia $f: ]a,b[  \to \mathbb{R}$ continua in $x_0$.\\
Supposto $\exists\ f^{\prime}_-(x_0),\ f^{\prime}_+(x_0)$ e $f^{\prime}_-(x_0) \neq f^{\prime}_+(x_0)$ allora il punto $(x_0, f(x_0)) \in \ ]a,b[ \times \mathbb{R}$ è:
\begin{enumerate}
\item[i)] \textbf{punto angoloso} se $f^{\prime}_-(x_0),\ f^{\prime}_+(x_0)$ o entrambe sono finite
\item[ii)] \textbf{cuspide} se $f^{\prime}_-(x_0) = +\infty$ e $f^{\prime}_+(x_0) = -\infty$ o viceversa
\item[iii)] \textbf{tangente verticale} se $f^{\prime}_-(x_0) = f^{\prime}_+(x_0) = +\infty$ o $-\infty$
\end{enumerate}