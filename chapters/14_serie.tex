\chapter{Serie}
\section{Definizione}
$$\sum_{k = 0}^{+\infty}a_k$$
Viene definita serie di termine generale $a_k$
\subsection{Convergenza e divergenza}
Sia $s_0 = a_0,\ s_1 = a_0 + a_1$ e in generale $\displaystyle s_n = \sum_{k = 0}^n a_k$.\\
La serie  è finita $\iff \displaystyle \lim_n s_n = s$ è finito.
$s$ è chiamato \textbf{somma della serie} e una serie finita è detta anche \textbf{convergente}.\\
In caso $\displaystyle \lim_n s_n = \infty$ la serie non è finita ed è detta \textbf{divergente}.\\
Convergenza e divergenza vengono dette il \textbf{Carattere} della serie.
\subsection{Osservazione - termini positivi}
Se $a_k \geq 0\ \forall\ k$ allora la serie sarà convergente oppure divergente a $+\infty$.
\subsubsection{Dimostrazione}
Essendo $s_n$ una somma di termini positivi essa è crescente
\subsection{Osservazione - numero finito di termini}
Un numero finito di termini non altera il carattere:\\
$$\sum_{k = 0}^{+\infty}a_k \text{ e } \sum_{k = n_0}^{+\infty}a_k$$ hanno lo stesso carattere

\section{Limite termine generale}
Se $\series{0} a_n$ è convergente $\implies a_n \xrightarrow[n]{}0$.\\
Quindi affinchè una serie sia convergente il fatto che il termine generale tenda a 0 è condizione necessaria, ma non sufficiente

\section{Serie notevoli}
\subsection{Serie geometrica}
\begin{equation}
\series{0} q^n = \begin{cases}
\frac{1}{1 + q} \text{ se } |q| < 1\\
+\infty \text{ se } q \geq 1\\
\text{IRREGOLARE se } q \leq -1\\
\end{cases}
\end{equation}
In particolare se la sommatoria è finita:
\begin{equation}
\displaystyle \sum_{k = 0}^{n} q^k = \frac{1 - q^{n + 1}}{1 - q}
\end{equation}