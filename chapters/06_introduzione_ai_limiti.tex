\chapter{Introduzione ai limiti}

\section{Distanza euclidea in $\mathbb{R}$}
$d(x,y) = | x-y | \quad \forall x,y \in \mathbb{R}$\\
La distanda $d$ da $x$ a $y$ è definita con il modulo della differenza $x-y$

Propietà:

\begin{enumerate}
	\item[i. ] $d(x,y) \geq 0 \quad \forall x,y \in \mathbb{R}$
	\item[ii. ] $d(x,y) = d(y,x) \quad \forall x,y \in \mathbb{R}$
	\item[iii. ] $d(x,y) \leq d(x,z) + d(z,y) \quad \forall x,y,z \in \mathbb{R} $  (disuguaglianza triangolare)
\end{enumerate}

\section{Definizione di intorno sferico}
dato $x_0 \in \mathbb{R} , \epsilon \in \mathbb{R}^+$, si dice intorno (sferico) di $x_0$, di raggio $\epsilon$, l'intervallo\\
$\mathrm{B}_\epsilon (x_0) = \{x \in \mathbb{R} : | x-x_0 |<\epsilon \} = (x_0 - \epsilon, x_0 + \epsilon)$\\

Disegno grafico TBD\\

L'insieme di tutti gli intorni di $x_0$ si definisce con\\
