\chapter{Limiti}
\section{Numeri reali estesi}
L'insieme dei numeri reali estesi $\overline{\mathbb{R}}$ si definisce come:
\begin{equation}
\overline{\mathbb{R}} \triangleq \mathbb{R} \cup \{+\infty,-\infty\}
\end{equation}
\subsection{Osservazioni}
Essendo $+\infty$ e $-\infty$ due simboli $\overline{\mathbb{R}}$ \textbf{non} è un insieme numerico

\section{Intorno}
\subsection{Intorno di un reale}
Fissati $x_0 \in \mathbb{R}$, $\epsilon >0$ chiameremo \textbf{intorno sferico} di centro $x_0$ e raggio $\epsilon$ l'intervallo:\\
$]x_0-\epsilon ,x_0+\epsilon[$ = $\{x \in \mathbb{R}:d(x,x_0)<\epsilon \}$ = $\{x \in \mathbb{R}: |x-x_0<\epsilon \}$
\subsection{Intorno di $+\infty$ e $-\infty$ }
Si dice \textbf{intorno sferico} di $+\infty$ qualunque semiretta t.c. $]M,+\infty[\ =\ \{x \in \mathbb{R}: x>M\}$ \\
Si dice \textbf{intorno sferico} di $-\infty$ qualunque semiretta t.c. $]-\infty ,M[\ =\ \{x \in \mathbb{R}: x<M\}$

\section{Estremi (relativi) di una funzione}
Siano $f:X \subseteq \mathbb{R} \rightarrow \mathbb{R}$, $x_0 \in X$, $I_{x_0}$ intorno di $x_0$:
\subsection{Minimo locale}
$x_0$ si dice \textbf{punto di minimo locale} (o relativo) di $f$ e $f(x_0)$ si dice \textbf{minimo locale} (o relativo) se:
\begin{equation}
\exists \ U \in I_{x_0}\ |\ f(x_0) \le f(x)\ \forall\ x \in U \cap X
\end{equation}
\subsection{Minimo locale ristretto}
$x_0$ si dice \textbf{punto di minimo locale ristretto} (o forte) di $f$ e $f(x_0)$ si dice \textbf{minimo locale ristretto} (o forte) se $x_0$ è un minimo locale e
\begin{equation}
f(x_0)<f(x)\ \forall x \neq x_0
\end{equation}
\subsection{Massimo locale}
$x_0$ si dice \textbf{punto di massimo locale} (o relativo) di $f$ e $f(x_0)$ si dice \textbf{massimo locale} (o relativo) se:
\begin{equation}
\exists \ U \in I_{x_0}\ |\ f(x_0) \geq f(x)\ \forall\ x \in U \cap X
\end{equation}
\subsection{Massimo locale ristretto}
$x_0$ si dice \textbf{punto di massimo locale ristretto} (o forte) di $f$ e $f(x_0)$ si dice \textbf{massimo locale ristretto} (o forte) se $x_0$ è un massimo locale e
\begin{equation}
f(x_0)>f(x)\ \forall x \neq x_0
\end{equation}

\section{Punto di accumulazione}
Sia $A \subseteq \mathbb{R}$, si dice che $x_0 \in \overline{\mathbb{R}}$ \`e un \textbf{punto di accumulazione} per $\mathbb{A}$ se:
\begin{equation}
\forall\ U \in I_{x_0}\ U \setminus \{x_0\} \cap \mathbb{A} \neq \emptyset
\end{equation}
ossia ogni intorno di $x_0$ contiene un punto di $\mathbb{A}$ diverso da $x_0$
\subsection{Punto di accumulazione sinistro}
Sia $x_0 \in \mathbb{R}$ si dice \textbf{punto di accumulazione sinistro} per $\mathbb{A}$ se $x_0$ \`e un punto di accumulazione per $\mathbb{A}\ \cap\ ]-\infty,x_0[$
\subsection{Punto di accumulazione sinistro}
Sia $x_0 \in \mathbb{R}$ si dice \textbf{punto di accumulazione destro} per $\mathbb{A}$ se $x_0$ \`e un punto di accumulazione per $\mathbb{A}\ \cap\ ]x_0,+\infty[$
\subsection{Punto isolato}
Sia $A \subseteq \mathbb{R}$, un punto $x_0 \in A$ che non \`e un punto di accumulazione per $A$ si dice \textbf{punto isolato}:	
\begin{equation}
\exists U \in I_{x_0}\ |\ U\ \cap\ \mathbb{A} = \{x_0\}
\end{equation}

\section{Limite}
Sia $\mathbb{X} \subseteq \mathbb{R}$, $f:\mathbb{X} \rightarrow \mathbb{R}$ una funzione, sia $x_0 \in \overline{\mathbb{R}}$ un punto di accumulazione per $\mathbb{X}$ \\
Allora $l \in \overline{\mathbb{R}}$ si dice \textbf{limite} per $f(x)$ tendente a $x_0$ e si scrive:
\begin{enumerate}
\item[•] $\displaystyle \lim_{x \to x_0}f(x) = l$
\item[•] $\displaystyle f(x) \xrightarrow[x \to x]{} l$
\end{enumerate}
\subsection{Limite sinitro e destro}
Sia $\mathbb{X} \subseteq \mathbb{R}$, $f:\mathbb{X} \rightarrow \mathbb{R}$ una funzione e $x_0 \in \mathbb{R}$ punto di accum. sinistro (o destro) per $\mathbb{X}$ \\
allora $l \in \overline{\mathbb{R}}$ si dice \textbf{limite sinistro o destro)} per $f(x)$ per $x \to x_0$, e si scrive \\
\begin{enumerate}
\item[•] $\displaystyle \lim_{x \to x_0^-}f(x) = l$ (sinistro)
\item[•] $\displaystyle \lim_{x \to x_0^+}f(x) = l$ (destro)\end{enumerate}
\subsection{Osservazione}
Se $\forall$ intorno $V$ di $l$, esiste un intorno (anche destro o sinistro) $U$ di $x_0$ t.c. $\forall x \in (U \cap X) \setminus \{x_0\}$ allora $f(x) \in V$

\section{Teorema dell'unicità del limite}
Se esiste il limite $\displaystyle \lim_{x \to x_0}f(x) = l$ allora esso è unico
\section{Teorema della limitatezza}
Se il limite $\displaystyle \lim_{x \to x_0}f(x) = l \in \mathbb{R}$ allora $f$ è limitata in un intorno di $x_0$
\section{Teorema della permanenza del segno}
Sia $x \subseteq \mathbb{R}$, $f:\mathbb{X} \rightarrow \mathbb{R}$, $x_0 \in \overline{\mathbb{R}}$ punto di accumulazione per $\mathbb{X}$ e $l \in \overline{\mathbb{R}}$
\begin{equation}
\begin{split}
\displaystyle \lim_{x \to x_0}f(x) = l > 0 \implies \exists \text{ un intorno } U \text{ di } x_0\ |\ f(x)>0\ \forall\ x \in U \cap \mathbb{X} \setminus \{x_0\} \\
\lim_{x \to x_0}f(x) = l < 0 \implies \exists \text{ un intorno } U \text{ di } x_0\ |\ f(x)<0\ \forall\ x \in U \cap \mathbb{X} \setminus \{x_0\}
\end{split}
\end{equation}
\section{Algebra dei limiti}
Siano $f, g$ due funzioni, $x_0 \in \overline{\mathbb{R}}$ punto di accumulazione per $\dom f \cap \dom g$
supponiamo $\displaystyle \lim_{x \to x_0}f(x) = l_f \in \mathbb{R}$ e $\displaystyle \lim_{x \to x_0}g(x) = l_g \in \mathbb{R}$ allora:
\begin{enumerate}
\item[a)]$f(x)\pm g(x) \rightarrow l_f \pm l_g$ per $x \to x_0$
\item[b)]$f(x)\cdot g(x) \rightarrow l_f \cdot l_g$ per $x \to x_0$
\item[c)]$\frac{f(x)}{g(x)} \rightarrow \frac{l_f}{l_g}$ per $x \to x_0$ ($l_g \neq 0$)
\end{enumerate}
\subsection{Estensione a $\overline{\mathbb{R}}$ }
Sia $x_0 \in \overline{\mathbb{R}}$ punto di accumulazione per $\dom f \cap \dom g$, allora per $x \to x_0$ si ha:
\begin{enumerate}
\item[d)]Se $f(x) \to +\infty$ e $g(x)$ limitata inferiormente in un intorno di $x_0$:\\
$f(x)+g(x) \longrightarrow +\infty$ \\
Se $f(x) \to -\infty$ e $g(x)$ limitata superiormente in un intorno di $x_0$:\\
$f(x)+g(x) \longrightarrow -\infty$
\item[e)]$f(x) \to +\infty$, $g(x) \to l_g \not=0$
\begin{itemize}
\item[i)]$f(x) \cdot g(x) = +\infty$ se $l_g>0$
\item[ii)]$f(x) \cdot g(x) = -\infty$ se $l_g<0$
\end{itemize}
\item[f)]Se $f(x) \to 0$ e $g(x)$ limitata in un intorno di $x_0$ :\\
$f(x)\cdot g(x) \longrightarrow 0$
\item[g)]$f(x) \to 0$, $f(x)>0$ in un intorno di $x_0$
\begin{itemize}
\item[i)]$\frac{1}{f(x)} \to +\infty$ se $f(x)>0$ in un intorno di $x_0$
\item[ii)]$\frac{1}{f(x)} \to -\infty$ se $f(x)<0$ in un intorno di $x_0$
\end{itemize}
\item[h)]Se $f(x) \to \pm \infty$: \\
$\frac{1}{f(x)} \longrightarrow 0$
\end{enumerate}
\section{Teorema del confronto (o dei due carabinieri)}
Siano $f,g,h: \mathbb{X} \subseteq \mathbb{R} \to \mathbb{R}$ e $x_0 \in \mathbb{\overline{R}}$ punto di accumulazione per $\mathbb{X}$\\
\begin{equation}
\begin{cases}
f(x) \leq g(x) \leq h(x) \text{ in un intorno } U \setminus \{x_0\} \text{ di } x_0 \in \dom f\\
\displaystyle \lim_{x \to x_0}f(x) = \lim_{x \to x_0}h(x) = l \in \mathbb{\overline{R}}
\end{cases}
\implies \exists \displaystyle \lim_{x \to x_0}g(x) = l
\end{equation}
\section{Limiti dx e sx di funzioni monotone}
\subsection{Monotonia crescente}
Se $f: \mathbb{X} \subseteq \mathbb{R} \to \mathbb{R}$ una funzione monotona crescente in $\mathbb{X}$
\begin{enumerate}
\item[•] $x_0 \in \mathbb{R}$ punto di accumulazione sinistro per $\mathbb{X} \implies\ \exists \displaystyle \lim_{x \to x_0^-}f(x) = \sup_{x \in \mathbb{X} \cap ]-\infty;x_0[}f$
\item[•] $x_0 \in \mathbb{R}$ punto di accumulazione destro per $\mathbb{X} \implies\ \exists \displaystyle \lim_{x \to x_0^+}f(x) = \inf_{x \in \mathbb{X} \cap ]x_0;+\infty[}f$
\end{enumerate} 
\subsection{Monotonia decrescente}
Se $f: \mathbb{X} \subseteq \mathbb{R} \to \mathbb{R}$ una funzione monotona decrescente in $\mathbb{X}$
\begin{enumerate}
\item[•] $x_0 \in \mathbb{R}$ punto di accumulazione sinistro per $\mathbb{X} \implies\ \exists \displaystyle \lim_{x \to x_0^-}f(x) = \inf_{x \in \mathbb{X} \cap ]-\infty;x_0[}f$
\item[•] $x_0 \in \mathbb{R}$ punto di accumulazione destro per $\mathbb{X} \implies\ \exists \displaystyle \lim_{x \to x_0^+}f(x) = \sup_{x \in \mathbb{X} \cap ]x_0;+\infty[}f$
\end{enumerate} 

\section{Continuità}
\subsection{Continuità in un punto}
$f:\mathbb{X} \subseteq \mathbb{R} \to \mathbb{R}$, $x \in \mathbb{X}$ è continua in $x_0 \in \mathbb{X}$ se:
\begin{enumerate}
\item[•]$x_0$ è un punto isolato di $\mathbb{X}$
\item[•]$
\begin{cases}
x_0 \text{ è un punto di accumulazione per }\mathbb{X}\\
\exists\ \displaystyle \lim_{x \to x_0}f(x) = f(x_0)
\end{cases}$
\end{enumerate}
\subsection{Continuità in un insieme}
$f:\mathbb{X} \subseteq \mathbb{R} \to \mathbb{R}$ si dice continua $\mathbb{A} \subseteq \mathbb{X}$ se è continua in tutti i punti di $\mathbb{A}$
\section{Limiti di funzioni composte}
Sia $f: \mathbb{X} \subseteq \mathbb{R} \to \mathbb{R}$, $g: \mathbb{Y} \subseteq \mathbb{R} \to \mathbb{R}$, $x_0 \in \overline{\mathbb{R}}$ punto di accumulazione per $\mathbb{X}$ e $y_0 \in \overline{\mathbb{R}}$ punto di accumulazione per $\mathbb{Y}$.
\begin{equation}
\begin{cases}
\displaystyle\lim_{x \to x_0}f(x)=y_0, f(x) \neq y_0\\
\displaystyle \lim_{y \to y_0} g(y)=k
\end{cases}
\implies \displaystyle\lim_{x \to x_0} g(f(x)) = \displaystyle\lim_{y \to y_0} g(y)=k
\end{equation}

\section{Gerarchia infiniti}
\begin{equation}
\begin{split}
|\log_b x|^{\alpha} \ll x^{\beta} \ll a^{x} \text{ per } x \to +\infty\\
\forall \alpha \in \mathbb{R}, \forall b>0, b \neq 1, \forall \beta >0, \forall a \ge 1
\end{split}
\end{equation}
In particolare per le successioni vale:

\begin{equation}
\begin{split}
\log_b n \ll n^{\alpha} \ll a^{n} \text{ per } x \to +\infty\\
\forall \alpha \in \mathbb{R}, \forall b>0, b \neq 1, \forall \beta > 0, \forall a \ge 1
\end{split}
\end{equation}

\section{Numero di Nepero $e$ }
Particolarmente interessante è studiare la successione $\{a_n=(1+ \frac{1}{n})^{n}, n \in \mathbb{N}, n \geq 1\}$. In particolare:
\begin{itemize}
\item[a)] $\{a_n\}$ \`e strettamente crescente
\item[b)] $\{a_n\}$ \`e limitata
\end{itemize}
Dal teorema di esistenza del limite per funzioni monotone, segue che esiste il limite, e in particolare è finito. Avendo il valore moltissime altre proprietà in matematica viene considerato una delle costanti fondamentali:
\begin{equation}
e \triangleq \displaystyle\lim_{n \to +\infty} \biggl(1+\frac{1}{n}\biggl)^n \text{      } (=\displaystyle\sup_{n} \{a_n\})
\end{equation}
$e \not \in \mathbb{Q}, e \in \mathbb{R}, e=2.71828...$
\subsection{Formula di Stirling}
$n! \sim (\frac{n}{e})^{n}\sqrt{2\pi n} \hspace{.6cm}$ per $n \to +\infty$ \\
\\
L'operatore $\sim$ tra due successioni è tale che $a_n \sim b_n \iff \displaystyle\lim_{n \to +\infty} \frac{a_n}{b_n} = 1$

\section{Divergenza}
Sia $f:\mathbb{X} \subseteq \mathbb{R} \to \mathbb{R}$, $x_0 \in \overline{\mathbb{R}}$ punto di accumulazione per $\mathbb{X}$
\begin{enumerate}
\item[i.]Se $\displaystyle\lim_{x \to x_0} f(x)=\pm \infty$, $f$ si dice \textbf{divergente} positiva (o negativa) oppure infinita (o un infinito) per $x \to x_0$
\item[ii.]Se $\displaystyle\lim_{x \to x_0} f(x)=0$, $f$ si dice infinitesima (o un infinitesimo) per $x \to x_0$
\end{enumerate}

\section{Confronto di infiniti}
Siano $f,g$ funzioni, $x_0 \in \overline{\mathbb{R}}$ punto di accumulazione per $\dom f \cap \dom g$. $f,g$ entrambe divergenti per $x \to x_0$ (e $g(x) \neq 0$ in un intorno di $x_0$)
\begin{enumerate}
\item[-]Se $\displaystyle \lim_{x \to x_0}\frac{f(x)}{g(x)} = 0$ si dice che $f$ è un infinito di ordine inferiore e $g$ di ordine superiore per $x \to x_0$
\item[-]Se $\displaystyle \lim_{x \to x_0}\frac{f(x)}{g(x)} = l \in \mathbb{R}$ si dice che $f$ e $g$ sono infiniti dello stesso ordine per $x \to x_0$\\
Se $l = 1$ si può anche scrivere $f(x) \sim g(x)$ per $x \to x_0$, cioè $f$ è asintotica a $g$ per $x \to x_0$ 
\end{enumerate}
Non è comunque detto che gli infiniti siano confrontabili.

\section{Confronto di infinitesimi}
Siano $f,g$ funzioni, $x_0 \in \overline{\mathbb{R}}$ punto di accumulazione per $\dom f \cap \dom g$. $f,g$ entrambe infinitesime per $x \to x_0$ (e $g(x) \neq 0$ in un intorno di $x_0$)
\begin{enumerate}
\item[-]Se $\displaystyle \lim_{x \to x_0}\frac{f(x)}{g(x)} = 0$ si dice che $f$ è un infinitesimo di ordine superiore e $g$ di ordine inferiore per $x \to x_0$
\item[-]Se $\displaystyle \lim_{x \to x_0}\frac{f(x)}{g(x)} = l \in \mathbb{R}$ si dice che $f$ e $g$ sono infinitesimi dello stesso ordine per $x \to x_0$\\
\end{enumerate}
Non è comunque detto che gli infinitesimi siano confrontabili.

\section{Notazione o-piccolo}
Siano $f,g$  funzioni infinitesime per $x \to x_0$ e $g(x) \neq 0$ in un intorno di $x_0$.\\
Se $\displaystyle \lim_{x \to x_0}\frac{f(x)}{g(x)} = 0$ è equivalente a scrivere $f(x) = o(g(x))$.\\
In particolare se $\displaystyle \lim_{x \to x_0}f(x) = 0$ è equivalente a $f(x) = o(1)$.
\subsection{Algebretta degli $o(1)$}
\begin{enumerate}
\item[i.] $ko(1) = o(1)\ \forall\ k \in \mathbb{R} \setminus \{0\}  (0 \times o(1) = 0$
\item[ii.] $o(1)\times o(1) = o(1)$
\item[iii.] $o(1) \pm o(1) = o(1)$ 
\end{enumerate} 
In particolare se $f(x) = o(|x-x_0|^{\alpha})$ si dice semplicemente che f è infinitesima di ordine superiore ad $\alpha$  per $x-x_0$
\subsection{Algebretta degli o-piccolo}
\begin{enumerate}
\item[i.] $ko(x^{\alpha}) = o(x^{\alpha})$
\item[ii.] $x^{\beta}o(x^{\alpha}) = o(x^{\alpha + \beta})$
\item[iii.] $o(x^{\beta})o(x^{\alpha}) = o(x^{\alpha + \beta})$
\item[iv.] $o(x^{\beta}) \pm o(x^{\alpha}) = o(x^{min(\alpha ,\beta)})$
\end{enumerate}

\section{Forme indeterminate}
Sia $f$ una funzione e $x_0$ un punto di accumulazione.\\
Nel caso $\displaystyle \lim_{x \to x_0}f(x)$ può assumere le seguenti forme che non appartengono ad $\overline{\mathbb{R}}$:
\begin{enumerate}
\item[•] $[+\infty - \infty]$
\item[•] $[\frac{\infty}{\infty}]$
\item[•] $[\frac{0}{0}]$
\item[•] $[0 \times \infty]$
\item[•] $[1^{\infty}]$
\item[•] $[0^{0}]$
\item[•] $[\infty^{\infty}]$
\end{enumerate}
Esse vengono chiamate forme indeterminate; è però spesso possibile trovare forme determinate per $\displaystyle \lim_{x \to x_0}g(x)$; con $g(x) = f(x)$. 

\section{Limiti notevoli}
Di seguito alcuni limiti notevoli utili da ricordare:
\begin{enumerate}
\item[a)] \begin{equation} \displaystyle \lim_{x \to 0}\frac{sen(x)}{x} = 1 \implies sen(x) \sim x
\end{equation}
Dimostrazione:\\
Per $x \to 0$ si ha che $|sin(x)| \leq |x| \leq |tg(x)|$. Dividendo per $|senx|$ e facendo il reciproco si ottiene $1 \geq |\frac{sin(x)}{x}| \geq |cos(x)|$ e essendo $\displaystyle \lim_{x \to 0}1 = \lim_{x \to 0}cos(x) = 1$ Per il teorema del confronto $\displaystyle \lim_{x \to 0}\frac{sen(x)}{x} = 1$
\item[b)] \begin{equation} \displaystyle \lim_{x \to 0}\frac{1 - cos(x)}{x^2} = \frac{1}{2} \implies 1 - cos(x) \sim \frac{1}{2}x^2
\end{equation}
\item[c)] \begin{equation} \displaystyle \lim_{x \to +\infty}\biggl (1 + \frac{1}{x} \biggl )^x = e
\end{equation}
\item[d)] \begin{equation} \displaystyle \lim_{x \to 0}\frac{ln(x+1)}{x} = 1 \implies ln(x + 1) \sim x
\end{equation}
\item[e)] \begin{equation} \displaystyle \lim_{x \to 0}\frac{e^x - 1}{x} = 1 \implies e^x - 1 \sim x
\end{equation}
\item[f)] \begin{equation} \displaystyle \lim_{x \to 0}\frac{(1 + x)^\alpha - 1}{x} = \alpha \implies (1 + x)^\alpha \sim 1 + \alpha x
\end{equation}
\end{enumerate}
